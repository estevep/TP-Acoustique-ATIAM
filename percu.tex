\documentclass[atiam, article]{rapport} % draft, phelma_black, phelma_normal, kit_de, kit_en, phelma_old

\doctitle{TP Percussions}
\title{TP Percussions\\Accord ou pas d'accord ?}
\titleheader{TP Percussions}
\titleone{}
\titletwo{Acoustique}
\titlethree{}
\author{Paul Estève, Étienne André} % Authors for the header
\autpage{ % Authors for the title page
  \begin{tabular}{l}
    Paul Estève \\
    Étienne André
  \end{tabular}
}
\supervisor[Encadrants : ]{Jean-Loïc Le Carrou\\Christophe Vergez
}
% \supervisorMail{}
\serie{ATIAM 2023/2024}
\date{\today}

\begin{document}

\maketitle

\section{Préparation du TP}

Afin d'avoir des mesures avec des amplitudes comparables, nous avons envisagé de laisser la mailloche pivoter entre nos doigts sous l'effet de la gravité, depuis le même angle et la même position. Cependant, il s'est avéré en séance que ces frappes n'étaient pas toujours aussi bien reproductibles que prévues, et que le rebond sur l'instrument n'est pas suffisant pour éviter une frappe multiple. Il faut donc rapidement rattraper la mailloche et vérifier que l'amplitude est cohérente.

Face à ces limitations, des frappes {\Large TODO description frappes}% TODO @etienne fais une mini description de comment on frappe (mouvement du poignet ? des doigts ? me souviens plus)
 répétées sur un même point seront effectuées avec un VU-mètre pour vérifier que l'amplitude max est proche.\footnote{Pour certains points de frappe (comme au centre de la timbale), on ressent expérimentalement qu'avec la même force, l'attaque est plus forte que pour les autres points.}

\section{Protocole de mesure}

Pour chacune des mesures, on utilise un microphone directif, relié à une carte son, enregistré sur Reaper. On analyse en temps réel le son capté avec un spectrogramme et un vu-mètre pour les questions d'amplitude (répétabilité de la frappe, éviter la saturation).

Pour le glockenspiel et le vibraphone, le micro est dirigé vers le milieu de la lame = point de frappe

Pour timbale, 1/4 du rayon -> modes antisymétriques

\section{Glockenspiel \& Vibraphone}

Pour ces deux instruments, on souhaite mesurer 



\section{Timbale}


\end{document}
