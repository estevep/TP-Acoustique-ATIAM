\documentclass[atiam, article]{rapport} % draft, phelma_black, phelma_normal, kit_de, kit_en, phelma_old

\doctitle{TP Cordes}
\title{TP Cordes\\Le ukulélé est-il une petite guitare ?}
\titleheader{TP Cordes}
\titleone{}
\titletwo{Acoustique}
\titlethree{}
\author{Paul Estève, Étienne André, Sébastien Li} % Authors for the header
\autpage{ % Authors for the title page
  \begin{tabular}{l}
    Paul Estève \\
    Étienne André\\
    Sébastien Li
  \end{tabular}
}
\supervisor[Encadrant : ]{Jean-Loïc Le Carrou}
% \supervisorMail{}
\serie{ATIAM 2023/2024}
\date{\today}

\bibliography{biblio}

\begin{document}

\maketitle

\section{Préparation du TP}

On utilise le modèle basse fréquence des instruments de type guitare présenté dans \cite{10.1121/1.384814}.

\begin{figure}
  \begin{center}
    \includegraphics{cordes/schema-modele.png}
  \end{center}
  \caption{Schéma du modèle simplifié en basse fréquence\cite{10.1121/1.384814}}
  \label{fig:schema-bf}
\end{figure}

La table d'harmonie est modélisée par un piston plan de surface équivalente $A$ et de masse $m_p$, lié à une raideur $k_p$. Celui-ci est couplé à l'ensemble {caisse de résonance + évent} modélisé par un résonateur de heimholtz de volume $V$ et de masse acoustique $m_a$ muni d'un trou de surface $S$.

La variation de pression à l'intérieur de la cavité $\Delta p$ est liée à la variation de volume $\Delta V$, dans l'hypothèse de compression adiabatique, par $\Delta p = \frac{-c^2 \rho}{V} \Delta V$ où $c$ est la vitesse du son et $\rho$ la masse volumique de l'air.

On ajoute au modèle les coefficients $R_p$ et $R_a$ représentant les résistances au mouvement du piston et de la masse acoustique.
Le déplacement du piston et de la masse d'air sont notés respectivement $x_p$ et $x_a$, relativement à leur position d'équilibre. Ils sont régis par le système d'équations:

\begin{equation}
  \begin{cases}
    m_p \ddot{x_p} = F - k_p x_p - R_p \dot{x_p} + A \Delta p  \\
    m_a \ddot{x_a} = S \Delta p - R_a \dot{x_a}    
  \end{cases}
\end{equation}

Dans la suite, on cherchera à identifier et comparer les paramètres équivalents de ce modèle sur la guitare et l'ukulélé, en suivant le protocole décrit dans \cite{10.1121/1.384814}.

\section{Protocole de mesure}

\begin{figure}
  \begin{center}
    \includegraphics[width=\textwidth/4]{cordes/guitare.jpg}
  \end{center}
  \caption{Guitare en situation de mesure avec masse ajoutée}
  \label{fig:photo-guitare}
\end{figure}
\begin{figure}
  \begin{center}
    \includegraphics[width=\textwidth/4]{cordes/ukulele.jpg}
  \end{center}
  \caption{Guitare en situation de mesure avec col}
  \label{fig:photo-ukulele}
\end{figure}

On suspend l'instrument à un bras de façon à ce qu'il ne soit pas amorti. On fait également glisser une bande de feutre dans les cordes, afin de les étouffer totalement. On peut voir les deux instruments en situation de mesure aux figures \ref{fig:photo-guitare}.et \ref{fig:photo-ukulele}.

Un accéléromètre est fixé au point de déplacement maximal de la table d'harmonie décrit dans \cite{fletcher2012physics}, juste au-dessous du chevalet pour la guitare et juste au-dessus pour l'ukulele, sur l'axe de symétrie de l'instrument. Muni d'un marteau de mesure, on frappe le plus proche possible de ce point, également sur l'axe de symétrie.

Un programme MATLAB calcule la fonction de transfert entre le marteau et l'accéléromètre.

Pour chacun des instruments, on effectue 4 mesures :
\begin{enumerate}
    \item Instrument sans modifications
    \item Instrument avec masse ajoutée, sous la forme d'un peu de pâte à modeler fixée sur l'axe de symétrie proche de l'accéléromètre
    \item Instrument avec évent bouché par du papier
    \item Instrument avec col de papier autour de l'évent à l'intérieur de la caisse de résonance.
\end{enumerate}

On veillera à ce que chaque mesure respecte les critères suivants:

\begin{itemize}
    \item La phase de la fonction de transfert doit être comprise entre $0$ et $-\pi$ pour assurer la colocalisation du point de frappe et de mesure de l'accélération.
    \item L'excitation doit être le plus proche possible d'un Dirac afin d'exciter une large gamme de fréquence, et surtout pour éviter les coups doubles.\item L'amplitude de l'excitation doit être suffisament grande pour obtenir des mesures représentatives, mais assez faible pour ne pas endommager le matériel.
\end{itemize}

\printbibliography


\end{document}
