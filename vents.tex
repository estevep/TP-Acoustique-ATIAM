\documentclass[atiam, article]{rapport} % draft, phelma_black, phelma_normal, kit_de, kit_en, phelma_old
\usepackage{booktabs}


\doctitle{TP Cordes}
\title{TP Cordes\\Le ukulélé est-il une petite guitare ?}
\titleheader{TP Cordes}
\titleone{}
\titletwo{Acoustique}
\titlethree{}
\author{Paul Estève, Étienne André, Sébastien Li} % Authors for the header
\autpage{ % Authors for the title page
  \begin{tabular}{l}
    Paul Estève \\
    Étienne André\\
    Sébastien Li
  \end{tabular}
}
\bibliography{biblio}
\supervisor[Encadrants : ]{Jean-Loïc Le Carrou\\Christophe Vergez}
% \supervisorMail{}
\serie{ATIAM 2023/2024}
\date{\today}

\begin{document}

\maketitle


\section{Préparation du TP}


\section{Protocole}

\section{Impédance d'entrée de la flûte}

\subsection{Etude théorique}

On souhaite étudier l'influence des pertes visco-thermiques sur le comportement dynamique de l'instrument. Dans un premier temps, une mesure d'impédance acoustique est réalisée dans différentes configurations, afin d'estimer les fréquences des modes 

Puisqu'on étudie un instrument en première approximation ouvert-ouvert, les fréquences de résonances correspondent aux maxima de l'admittance d'entrée/de sortie et aux minima de l'impédance d'entrée/de sortie.

Notre capteur ayant une surface place plan, on y place l'extrémité "de sortie" de la flûte (opposée à l'embouchure) et on mesure donc l'impédance de sortie de l'instrument.

Le modèle utilisé est celui de Auvray et al. \cite{Auvray2011}. Il permet de modéliser l'impédance de la flûte après le bec, à l'entrée du tuyau, là où le champ acoustique interagit sur le jet.
$$Y_e=Y_c \frac{\cosh \left(\Gamma l+\eta_l\right) \cos \eta_m}{\sinh \left(\Gamma l+\eta_l+\eta_m\right)}$$

avec $Y_c = \frac{S}{\rho c S_m}$, $S_m$ la section de la bouche, $l$ la longueur du tuyau, $\Gamma$ le nombre d'onde complexe et $\eta_i = \arg \tanh\left(\frac{Z_i}{Z_c}\right)$.

En utilisant la méthode de l'impédance ramenée, on peut en déduire une formule théorique de l'impédance de sortie.

Si $$\begin{pmatrix}
    P_s\\U_s
\end{pmatrix} =
M
\begin{pmatrix}
    P_e\\U_e
\end{pmatrix}$$

Avec $M = \begin{pmatrix}A & B\\C&D\end{pmatrix}$, on a $Y_s = \frac{U_s}{P_s} = \frac{C + D Y_e}{C Z_e + D}$. La matrice $M$ est définie comme

\begin{equation}
    M = \begin{pmatrix}
        \cosh \Gamma l & Z_c \sinh \Gamma l \\
        Z_c^{-1} \sinh \Gamma l & \cosh \Gamma l
    \end{pmatrix}
    \label{eq:mat_trans_cylindre}
\end{equation}

pour un cylindre de longueur $l$, avec $Z_c = \sqrt{\frac{Z_v}{Y_t}}$ pour tenir compte des pertes visqueuses et thermiques comme indiqué durant le cours\cite{Vergez4}.

On vérifie dans un premier temps que les mesures d'impédances sont cohérentes 

\begin{table}[H]
    \centering
    \begin{tabular}{ccc}
\toprule
$f_n$ & $Q_n$ & $Y_n$ \\ \midrule
180 & 8 & 1032.07 \\
898 & 75 & 3096.22 \\
1810 & 70 & 3440.25 \\
2720 & 50 & 4128.30 \\
3630 & 50 & 4128.30 \\
\bottomrule\\
    \end{tabular}
    \caption{Paramètres du modèle pour le cylindre seul}
    \label{tab:param_cylindre}
\end{table}

\begin{table}[H]
    \centering
    \begin{tabular}{ccc}
\toprule
$f_n$ & $Q_n$ & $Y_n$ \\ \midrule
190 & 5 & 1032.07 \\
877 & 13 & 2545.78 \\
1770 & 12 & 3096.22 \\
2710 & 17 & 4472.32 \\
3555 & 20 & 2752.20 \\
\bottomrule\\
    \end{tabular}
    \caption{Paramètres du modèle pour le cylindre avec stack}
    \label{tab:param_cylindre+stack}
\end{table}




Pour chacun des instruments, on effectue 4 mesures, cordes étouffées :
\begin{enumerate}
    \item Instrument sans modifications
    \item Instrument avec masse ajoutée
    \item Instrument avec évent bouché
    \item Instrument avec col
\end{enumerate}

Celles-ci peuvent être répétées pour avoir au moins une mesure qui respecte les critères suivants :

\begin{itemize}
    \item Phase de la FRF comprise entre $0$ et $-\pi$ pour assurer la colocalisation du point de frappe et de mesure de l'accélération.
    \item Frappe au plus proche possible d'un Dirac $\Leftrightarrow$ Excitation avec un niveau d'énergie significatif et assez constant pour la gamme de fréquences étudiée.
    \item Pas de présence de pic avec une largeur dans la 
\end{itemize}

\printbibliography

\end{document}
