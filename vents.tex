\documentclass[atiam, article]{rapport} % draft, phelma_black, phelma_normal, kit_de, kit_en, phelma_old

\doctitle{TP Cordes}
\title{TP Cordes\\Le ukulélé est-il une petite guitare ?}
\titleheader{TP Cordes}
\titleone{}
\titletwo{Acoustique}
\titlethree{}
\author{Paul Estève, Étienne André, Sébastien Li} % Authors for the header
\autpage{ % Authors for the title page
  \begin{tabular}{l}
    Paul Estève \\
    Étienne André\\
    Sébastien Li
  \end{tabular}
}
\supervisor[Encadrant : ]{Jean-Loïc Le Carrou}
% \supervisorMail{}
\serie{ATIAM 2023/2024}
\date{\today}

\begin{document}

\maketitle

\section{Preparation du TD}

\section{Protocole}

Pour chacun des instruments, on effectue 4 mesures, cordes étouffées :
\begin{enumerate}
    \item Instrument sans modifications
    \item Instrument avec masse ajoutée
    \item Instrument avec évent bouché
    \item Instrument avec col
\end{enumerate}

Celles-ci peuvent être répétées pour avoir au moins une mesure qui respecte les critères suivants :

\begin{itemize}
    \item Phase de la FRF comprise entre $0$ et $-\pi$ pour assurer la colocalisation du point de frappe et de mesure de l'accélération.
    \item Frappe au plus proche possible d'un Dirac $\Leftrightarrow$ Excitation avec un niveau d'énergie significatif et assez constant pour la gamme de fréquences étudiée.
    \item Pas de présence de pic avec une largeur dans la 
\end{itemize}

\end{document}
